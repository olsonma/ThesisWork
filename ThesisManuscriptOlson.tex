\listfiles
\documentclass[12pt]{report}\usepackage[]{graphicx}\usepackage[]{color}
%% maxwidth is the original width if it is less than linewidth
%% otherwise use linewidth (to make sure the graphics do not exceed the margin)
\makeatletter
\def\maxwidth{ %
  \ifdim\Gin@nat@width>\linewidth
    \linewidth
  \else
    \Gin@nat@width
  \fi
}
\makeatother

\definecolor{fgcolor}{rgb}{0.345, 0.345, 0.345}
\newcommand{\hlnum}[1]{\textcolor[rgb]{0.686,0.059,0.569}{#1}}%
\newcommand{\hlstr}[1]{\textcolor[rgb]{0.192,0.494,0.8}{#1}}%
\newcommand{\hlcom}[1]{\textcolor[rgb]{0.678,0.584,0.686}{\textit{#1}}}%
\newcommand{\hlopt}[1]{\textcolor[rgb]{0,0,0}{#1}}%
\newcommand{\hlstd}[1]{\textcolor[rgb]{0.345,0.345,0.345}{#1}}%
\newcommand{\hlkwa}[1]{\textcolor[rgb]{0.161,0.373,0.58}{\textbf{#1}}}%
\newcommand{\hlkwb}[1]{\textcolor[rgb]{0.69,0.353,0.396}{#1}}%
\newcommand{\hlkwc}[1]{\textcolor[rgb]{0.333,0.667,0.333}{#1}}%
\newcommand{\hlkwd}[1]{\textcolor[rgb]{0.737,0.353,0.396}{\textbf{#1}}}%
\let\hlipl\hlkwb

\usepackage{framed}
\makeatletter
\newenvironment{kframe}{%
 \def\at@end@of@kframe{}%
 \ifinner\ifhmode%
  \def\at@end@of@kframe{\end{minipage}}%
  \begin{minipage}{\columnwidth}%
 \fi\fi%
 \def\FrameCommand##1{\hskip\@totalleftmargin \hskip-\fboxsep
 \colorbox{shadecolor}{##1}\hskip-\fboxsep
     % There is no \\@totalrightmargin, so:
     \hskip-\linewidth \hskip-\@totalleftmargin \hskip\columnwidth}%
 \MakeFramed {\advance\hsize-\width
   \@totalleftmargin\z@ \linewidth\hsize
   \@setminipage}}%
 {\par\unskip\endMakeFramed%
 \at@end@of@kframe}
\makeatother

\definecolor{shadecolor}{rgb}{.97, .97, .97}
\definecolor{messagecolor}{rgb}{0, 0, 0}
\definecolor{warningcolor}{rgb}{1, 0, 1}
\definecolor{errorcolor}{rgb}{1, 0, 0}
\newenvironment{knitrout}{}{} % an empty environment to be redefined in TeX

\usepackage{alltt}

\usepackage[intoc]{nomencl}
\textwidth=6in \oddsidemargin=0.5in \topmargin=-0.5in
\textheight=9in  % 9in must include page numbers
\textfloatsep = 0.4in 
\addtocontents{toc}{\vspace{0.4in} \protect \hfill Page\endgraf} 
\addtocontents{lof}{\vspace{0.2in} \hspace{0.13in} \ Figure \protect \hfill Page\endgraf} \addtocontents{lot}{\vspace{0.2in} \hspace{0.13in} \ Table \protect\hfill Page\endgraf}

%%%%%%%%%%%%%%%%%%%%%%%%%%%%%%%
%-------------- USE PACKAGE-----------------------------------------%
%%%%%%%%%%%%%%%%%%%%%%%%%%%%%%%

%\usepackage[natbibapa]{apacite}

\usepackage{amsmath}
\usepackage{mathtools}
\usepackage{graphicx}
\usepackage{multirow}
\usepackage[font=singlespacing]{caption}
%\usepackage{caption}
%\captionsetup{font=scriptsize}
\captionsetup{font=footnotesize}
\usepackage[nottoc,notlof,notlot]{tocbibind}
\renewcommand\bibname{REFERENCES}
%\usepackage[backend=bibtex,style=verbose-trad2]{biblatex}
%\bibliography{/Users/mollyolson/Documents/Vanderbilt/Masters_Thesis/ThesisWork/thesisBibliography}		
%\usepackage{biblatex}

%\usepackage{cite}


\usepackage{setspace}
\usepackage{titlesec}
\usepackage{tgschola}
\usepackage{color}
\usepackage[left=1.5in,right=1in,top=1in,bottom=1in]{geometry}
\usepackage[table]{xcolor}
 \usepackage{amsfonts}
 \usepackage{amsmath}
 \usepackage{amsbsy,bm}
 \usepackage{amssymb}
\usepackage{graphicx}
 \usepackage{setspace}
 \usepackage{rotating}
 \usepackage{float}
 \usepackage{stmaryrd}
 \usepackage{multirow}
 \usepackage{color}
 \usepackage{soul}
 \usepackage{caption}
\usepackage{eepic}
\usepackage{colortbl}
%\usepackage[numbers]{natbib}
%\usepackage{natbib}
%\setcitestyle{citesep={;}, aysep={,}}
\newcommand\harvardand{\&}
%\renewcommand{\bibname}{References}



\usepackage{multirow}
\usepackage{setspace}
\usepackage{indentfirst}
\usepackage{titlesec}
\usepackage{subfig}
\usepackage[mathscr]{euscript}
\usepackage[titletoc,title]{appendix}
\usepackage[titletoc]{appendix}
\usepackage[tocgraduated]{tocstyle}

\usepackage{textcomp}
\usepackage{array}
\usepackage{listings}
\usepackage{setspace}
\usepackage{mathptmx}
\usepackage{colortbl}
\usepackage{graphicx}
\usepackage{amssymb, amsmath}
\usepackage{subfig}
\usepackage{epsfig}
\usepackage{times}
\usepackage{float}
\usepackage{rotating}
\usepackage{makeidx}
\usepackage{url}
\usepackage{multirow}
\usepackage{booktabs}
\usepackage[subfigure, titles]{tocloft}
\usepackage[hidelinks]{hyperref}

\usepackage{acronym}
\usepackage{datetime}
\usepackage{algorithm}
\usepackage{algorithmic}
\usepackage{url, hyperref}
%\usepackage{cleveref}
\renewcommand{\nomname}{LIST OF ABBREVIATIONS}
\makenomenclature
\graphicspath{{images/}}
\DeclareGraphicsExtensions{.pdf,.jpeg,.png,.PNG, .eps, .tiff}

\urlstyle{same}

\usepackage{makecell}
\usepackage{titletoc}
\usepackage{appendix}
\usepackage[nottoc]{tocbibind}
\setcounter{secnumdepth}{7}
\setcounter{tocdepth}{7}
\usepackage{lscape}

\DeclarePairedDelimiter\ceil{\lceil}{\rceil}
\DeclarePairedDelimiter\floor{\lfloor}{\rfloor}

%%%%%%%%%%%%%%%%%%%%%%%%%%%%%%%
%-------------- NEW COMMANDS-------------------------------------%
%%%%%%%%%%%%%%%%%%%%%%%%%%%%%%%
%new chapter/section and subsection commands
\newcommand{\hsuchapter}[1]{\chapter*{#1} \addcontentsline{toc}{chapter}{#1} } 
\newcommand{\hsusection}[1]{\section*{#1} \addcontentsline{toc}{section}{#1} } 
\newcommand{\hsusubsection}[1]{\subsection*{#1} \addcontentsline{toc}{subsection}{#1} } 

%%%%%%%Configure Table of Contents%%%%%%%%%%%%
\renewcommand{\contentsname}{TABLE OF CONTENTS}
\renewcommand{\cftchapfont}{\normalfont}
\renewcommand{\cftchappagefont}{\normalfont}
\renewcommand{\cftchapleader}{\cftdotfill{\cftdotsep}}

%%%%%%%Configure List of Figures%%%%%%%%%%%%
\renewcommand{\listfigurename}{LIST OF FIGURES}
\setlength{\cftbeforefigskip}{0.2in}

%%%%%%%Configure List of Tables%%%%%%%%%%%%
\renewcommand{\listtablename}{LIST OF TABLES}
\setlength{\cftbeforetabskip}{0.2in}

%%%%%% Configure ABSTRACT %%%%%%
\usepackage{abstract}
\renewcommand{\abstractname}{ABSTRACT}




%%%%%%%Configure Bibliography%%%%%%%%%%%%
\renewcommand{\bibname}{ \texorpdfstring{{REFERENCES\vspace{10mm}}}{REFERENCES}   }

%%%%%%%%%%%%%%%%%%%%%%%%%%%%%%%
%-------------- CONFIGURE CHAPTER HEADINGS--------------%
%%%%%%%%%%%%%%%%%%%%%%%%%%%%%%%



\makeatletter
\def\@makechapterhead#1{
  {\parindent \z@ %\centering
    \huge \fontfamily{qcs}\selectfont
    \ifnum \c@secnumdepth >\m@ne
      \if@mainmatter
        \@chapapp\space \thechapter
        \par\nobreak
        \vskip 20\p@
      \fi
    \fi
    \interlinepenalty\@M
    #1\par\nobreak
    \vskip 40\p@
  }}
\def\@schapter#1{\if@twocolumn
                   \@topnewpage[\@makeschapterhead{#1}]%
                 \else
                   \@makeschapterhead{#1}%
                   \@afterheading
                 \fi}
\def\@makeschapterhead#1{
  {\parindent \z@ \centering
    \large
    \interlinepenalty\@M
    #1\par\nobreak
    \vskip 10\p@
  }}




%%% this configure the linespace in the table of content
%%% code is complicated and ugly but it works
\newlength{\li}\setlength{\li}{14.48pt}
\newlength{\di}\setlength{\di}{-3.5mm}
\def\@chapter[#1]#2{\ifnum \c@secnumdepth >\m@ne
      \refstepcounter{chapter}%
      \typeout{\@chapapp\space\thechapter.}%
      \addcontentsline{toc}{chapter}{\numberline{\thechapter}#1}
         %{\protect\numberline{\thechapter}\uppercase{#1}}%
      \addtocontents{toc}{\protect\vspace{\li}}%
  \else
      %\addcontentsline{toc}{chapter}{\uppercase{#1}}%
      \addcontentsline{toc}{chapter}{#1}
      \addtocontents{toc}{\protect\vspace{\li}}%
  \fi
  \chaptermark{#1}%
  \if@twocolumn
      \@topnewpage[\@makechapterhead{#2}]%
  \else
      \@makechapterhead{#2}%
      \@afterheading
 \fi}


\renewcommand\chapter{\addtocontents{toc}{\protect\addvspace{\li}}%
  \if@openright\cleardoublepage\else\clearpage\fi
  \thispagestyle{plain}%
  \global\@topnum\z@
  \@afterindentfalse
  \secdef\@chapter\@schapter}

%%%%%%%%%%%%%%%%%%%%%%%%%%%%%%%%%%%%%%%%%%%%%%%%%%%%%%%%%%%%%%
%-----------CONFIGURE SECTION HEADINGS------------------%
%%%%%%%%%%%%%%%%%%%%%%%%%%%%%%%%%%%%%%%%%%%%%%%%%%%%%%%%%%%%%%
\renewcommand\section{ \@startsection {section}{1}{\z@}%
                                   {-3.5ex \@plus -1ex \@minus -.2ex}%
                                   {2.3ex \@plus.2ex}%
                                   {\centering\large\fontfamily{qcs}\selectfont}}


                    
%%%%%%%%%%%%%%%%%%%%%%%%%%%%%%%%%%%%%%%%%%%%%%%%%%%%%%%%%%%%%%
%-------------CONFIGURE SUBSECTION HEADINGS- --------%
%%%%%%%%%%%%%%%%%%%%%%%%%%%%%%%%%%%%%%%%%%%%%%%%%%%%%%%%%%%%%%
\renewcommand\subsection{\@startsection {subsection}{2}{\z@}%
                                   {-3.5ex \@plus -1ex \@minus -.2ex}%
                                   {2.3ex \@plus.2ex}%
                                   {\noindent \large \fontfamily{qcs}\selectfont }}
                                  
      
                                   

%%%%%%%Sub-Sub-Section's Not  Supported%%%%%%%%%%%%

%%%%%%%%%%%%%%%%%%%%%%%%%%%%%%%%%%%%%%%%%%%%%%%%%%%%%%%%%%%%%%
%-------CONFIGURE TABLE OF CONTENTS HEADING------%
%%%%%%%%%%%%%%%%%%%%%%%%%%%%%%%%%%%%%%%%%%%%%%%%%%%%%%%%%%%%%%
\renewcommand{\@cftmaketoctitle}{
  \chapter*{\contentsname}
  \addcontentsline{toc}{chapter}{TABLE OF CONTENTS}} 

%%%%%%%%%%%%%%%%%%%%%%%%%%%%%%%%%%%%%%%%%%%%%%%%%%%%%%%%%%%%%%
%------CONFIGURE LIST OF FIGURES HEADING------------%
%%%%%%%%%%%%%%%%%%%%%%%%%%%%%%%%%%%%%%%%%%%%%%%%%%%%%%%%%%%%%%
\renewcommand{\@cftmakeloftitle}{
  \chapter*{\listfigurename}
  Figure \hfill Page
  \addcontentsline{toc}{chapter}{LIST OF FIGURES} } 
  
%%%%%%%%%%%%%%%%%%%%%%%%%%%%%%%%%%%%%%%%%%%%%%%%%%%%%%%%%%%%%%
%--------CONFIGURE LIST OF TABLES HEADING-------------%
%%%%%%%%%%%%%%%%%%%%%%%%%%%%%%%%%%%%%%%%%%%%%%%%%%%%%%%%%%%%%%
\renewcommand{\@cftmakelottitle}{
  \chapter*{\listtablename}
   Table \hfill Page
   \addcontentsline{toc}{chapter}{LIST OF TABLES} }  

\makeatother

\setcounter{section}{-1}     

%%%%%%%%%%%%%%%%%%%%%%%%%%%%%%%%%%%%%%%%%%%%%%%%%%%%%%%%%%%%%%
%---------------------------NEW COMMANDS-------------------------%
%%%%%%%%%%%%%%%%%%%%%%%%%%%%%%%%%%%%%%%%%%%%%%%%%%%%%%%%%%%%%%
\newcommand{\etal}{\emph{et al.}}
\newcommand{\leftsup}[2]{{\vphantom{#2}}^{#1}{#2}}
\newcommand{\leftsub}[2]{{\vphantom{#2}}_{#1}{#2}}
\newcommand{\leftsupsub}[3]{{\vphantom{#3}}^{#1}_{#2}{#3}}

\DeclareMathOperator*{\assembly}{\textbf{\Large A} }

\definecolor{lightblue}{rgb}{.90,.95,1} 
\newcommand{\hllb}[1]{
	\sethlcolor{lightblue}
	\hl{#1}
	\sethlcolor{yellow}
	}

\newcommand{\hlc}[2][yellow]{{\sethlcolor{#1}\hl{#2}} }

%%%%%%%%%%%%%%%%%%%%%%%%%%%%%%%%%%%%%%%%%%%%%%%%%%%%%%%%%%%%%%
%--------------DEFINE FLOATS----------------------------------------%
%%%%%%%%%%%%%%%%%%%%%%%%%%%%%%%%%%%%%%%%%%%%%%%%%%%%%%%%%%%%%%
 \floatstyle{plain}
 \newfloat{Box}{h}{lob}
 \newcommand{\boxedtext}[3]{
 	\begin{Box} \caption{\small{#1}}
	\hspace{1.cm}
	\fbox{\begin{minipage}[c]{0.85\linewidth} 
	
	\small{#2}
       
       \end{minipage}}
       
       \label{#3}
       \end{Box}
  }
\IfFileExists{upquote.sty}{\usepackage{upquote}}{}
\begin{document}

%\pagestyle{myheadings} \markright{\today}
%%%%%%%%%%%%%%%%%%%%%%%%%%%%%%%%%%%%%%%%%%%%%%%%%%%%%%%%%%%%%%
%-------------- MAKE TITLE CHANGES HERE---------------------%
%%%%%%%%%%%%%%%%%%%%%%%%%%%%%%%%%%%%%%%%%%%%%%%%%%%%%%%%%%%%%%
\pagenumbering{alph}

\begin{titlepage}
\thispagestyle{empty}\enlargethispage{\the\footskip}%
\begin{center}
	{\setstretch{1.66} {Working Title: A Comparison of Approaches for Unplanned Sample Sizes in Phase II Clinical Trials}\par }%
	\vskip.4in
	By
	\vskip .3in
	{Molly Olson}
	\vskip .3in
	
	\begin{doublespace}
	Thesis\\
		Submitted to the Faculty of the \\
		Graduate School of Vanderbilt University \\
		in partial fulfillment of the requirements \\
		for the degree of \\ [.1in]
	\end{doublespace}
	
	\MakeUppercase{MASTER OF SCIENCE} \\[.1in]
	in \\[.1in]
	{Biostatistics} \\[.25in]
	May, 2017 \\[.25in]
	Nashville, Tennessee
	\vskip .5in
%\end{center}
%%%Uncomment for Signatures%%%
%Approved: \hskip 2.9in Date:\\[1.2em]
%\rule{3.5in}{.5pt} \hskip 0.1in \rule{2in}{.5pt} \\[.01in]
%Professor John D. Doe \\[.14in]
%\rule{3.5in}{.5pt} \hskip 0.1in \rule{2in}{.5pt}  \\[.01in]
%Professor John D. Doe \\[.14in]
%\rule{3.5in}{.5pt} \hskip 0.1in \rule{2in}{.5pt} \\[.01in]
%Professor John D. Doe \\[.14in]
%\rule{3.5in}{.5pt} \hskip 0.1in \rule{2in}{.5pt} \\[.01in]
%Professor John D. Doe \\[.14in]
%\\[.14in]
%%%%%%%%%%%%%%
%%%%%%Uncomment  for Approved Names%%%%%%
\begin{doublespace}
Approved (in progress):\\
Tatsuki Koyama , Ph.D. \\
Jeffrey Blume , Ph.D. \\
\end{doublespace}
%%%%%%%%%%%%%%%%%%%%%%%%%%%%%%%%%%%%%%%
\end{center}
\end{titlepage}
 
\doublespacing
\pagenumbering{roman} \setcounter{page}{2}

%\chapter*{The dedication page is optional. If you don't use it, please delete it.}
%\addcontentsline{toc}{chapter}{DEDICATION}
%\vspace{7mm}

%%%%%%%%%%%%%%%%%%%%%%%%%%%%%%%%%%%%%%%%%%%%%%%%%%%%%%%%%%%%%%
%--------------ACKNOWLEDGEMENTS----------- -----------------%
%%%%%%%%%%%%%%%%%%%%%%%%%%%%%%%%%%%%%%%%%%%%%%%%%%%%%%%%%%%%%%
\chapter*{ACKNOWLEDGMENTS}
\addcontentsline{toc}{chapter}{ACKNOWLEDGMENTS}
\vspace{7mm}


%%%%%%%%%%%%%%%%%%%%%%%%%%%%%%%%%%%%%%%%%%%%%%%%%%%%%%%%%%%%%%
%-------------- BEGIN TABLE OF CONTENTS---------------------%
%%%%%%%%%%%%%%%%%%%%%%%%%%%%%%%%%%%%%%%%%%%%%%%%%%%%%%%%%%%%%%
\begin{singlespace}
\tableofcontents
\newpage
\addcontentsline{toc}{chapter}{\listtablename}
\end{singlespace}

%%%%%%%%%%%%%%%%%%%%%%%%%%%%%%%%%%%%%%%%%%%%%%%%%%%%%%%%%%%%%%
%--------------BEGIN LIST OF TABLES------------------------------%
%%%%%%%%%%%%%%%%%%%%%%%%%%%%%%%%%%%%%%%%%%%%%%%%%%%%%%%%%%%%%%
\listoftables

%%%%%%%%%%%%%%%%%%%%%%%%%%%%%%%%%%%%%%%%%%%%%%%%%%%%%%%%%%%%%%
%--------------BEGIN LIST OF FIGURES----------------------------%
%%%%%%%%%%%%%%%%%%%%%%%%%%%%%%%%%%%%%%%%%%%%%%%%%%%%%%%%%%%%%%
\newpage
\addcontentsline{toc}{chapter}{\listfigurename}
\listoffigures
\newpage
%%%%%%%%%%%%%%%%%%%%%%%%%%%%%%%%%%%%%%%%%%%%%%%%%%%%%%%%%%%%%%
%-------------- ABSTRACT-----------------------------------------------%
%%%%%%%%%%%%%%%%%%%%%%%%%%%%%%%%%%%%%%%%%%%%%%%%%%%%%%%%%%%%%%
\addcontentsline{toc}{chapter}{ABSTRACT}
\chapter*{ABSTRACT}

\vspace{7mm}
In this thesis, we develop .....
\newpage

\normalsize
\doublespacing
\pagenumbering{arabic}
\setcounter{page}{1}
%%%%%%%%%%%%%%%%%%%%%%%%%%%%%%%%%%%%%%%%%%%%%%%%%%%%%%%%%%%%%%%%%%%%%%%%%%%%%%%%%%%%%%%%%%%%%%%%%%%%%%%%%%%%%%%%%%
%%%%%%%%%%%%%%%%%%%%%%%%%%%%%%%%%%%%%%%%%%%%%%%%%%%%%%%%%%%%%%%%%%%%%%%%%%%%%%%%%%%%%%%%%%%%%%%%%%%%%%%%%%%%%%%%%%
%%%%%%%%%%%%%%%%%%%%%%%%%%%%%%%%%%%%%%%%%%%%%%%%%%%%%%%%%%%%%%%%%%%%%%%%%%%%%%%%%%%%%%%%%%%%%%%%%%%%%%%%%%%%%%%%%%
%%%%%%%%%%%%%%%%%%%%%%%%%%%%%%%%%%%%%%%%%%%%%%%%%%%%%%%%%%%%%%%%%%%%%%%%%%%%%%%%%%%%%%%%%%%%%%%%%%%%%%%%%%%%%%%%%%
%% -----------------------------------WRITING STARTS HERE ------------------------------------------------------%%
%%%%%%%%%%%%%%%%%%%%%%%%%%%%%%%%%%%%%%%%%%%%%%%%%%%%%%%%%%%%%%%%%%%%%%%%%%%%%%%%%%%%%%%%%%%%%%%%%%%%%%%%%%%%%%%%%%
%%%%%%%%%%%%%%%%%%%%%%%%%%%%%%%%%%%%%%%%%%%%%%%%%%%%%%%%%%%%%%%%%%%%%%%%%%%%%%%%%%%%%%%%%%%%%%%%%%%%%%%%%%%%%%%%%%
%%%%%%%%%%%%%%%%%%%%%%%%%%%%%%%%%%%%%%%%%%%%%%%%%%%%%%%%%%%%%%%%%%%%%%%%%%%%%%%%%%%%%%%%%%%%%%%%%%%%%%%%%%%%%%%%%%
%%%%%%%%%%%%%%%%%%%%%%%%%%%%%%%%%%%%%%%%%%%%%%%%%%%%%%%%%%%%%%%%%%%%%%%%%%%%%%%%%%%%%%%%%%%%%%%%%%%%%%%%%%%%%%%%%%

%---------------------------------------------------------------------------------------------------------------%
%------------------------------------------------CHAPTER1------------------------------------------------%
%---------------------------------------------------------------------------------------------------------------%


%%%%%%%%%%%%%%%%%%%%%%%%%%%%%%%%%%%%%%%%%%%%%%%%%%%%%%%%%%%%%%
%------------Introduction------------------------------------%
%%%%%%%%%%%%%%%%%%%%%%%%%%%%%%%%%%%%%%%%%%%%%%%%%%%%%%%%%%%%%%
\cftlocalchange{toc}{450pt}{0cm}
\cftaddtitleline{toc}{chapter}{Chapter}{}
\cftlocalchange{toc}{1.55em}{2.55em}
\chapter{Introduction}
\vspace{-7mm}
The introduction will talk about the motivation for the thesis. Introduce some examples of when we would need such methods. 
\newline
\newline

Oncology phase II clinical trials are often used to evaluate the initial effect of a new regimen to determine if to warrant further study in a phase III clinical trial \cite{Porcher, Simon, Koyama}. Simon's two-stage design \cite{Simon} is a commonly used design in specifying sample sizes and critical values in phase II oncology clinical trials. Koyama and Chen \cite{Koyama} point out that it is common for actual sample sizes of these phase II trials to differ than the planned, pre-specified sample sizes. This could happen because of unanticipated accruement speed, drop-out rates are unexpected, and often multi-center trials can be slow in sharing information ... Currently, when unplanned sample sizes occur, it is common practice to treat the attained sample sizes as planned. Though, when acheived sample sizes differ from planned, hypothesis testing using the attained sample sizes as if they were planned is not valid and hypothesis testing in these cases is not straightforward \cite{Porcher, Koyama}. Because of these reasons, extensions of Simon's design for hypothesis testing with unplanned sample sizes is important. 
\newline
\textit{Talk about examples here}
\newline
There have been many attempts to develop Frequentist methods that handle unplanned sample sizes in the second stage while using the planned stage I sample size, but my literature review found that there were only few Frequentist methods to handle unplanned sample sizes in both stage I and stage II. Likelihood based designs, that are able to be an extension of Simon's design, offer a nice solution to this problem because these designs offer flexibility in sample size without inflation of type I error.  In this paper, we discuss the different methods for Simon's design when the attained stage II sample size is different than planned and when attained sample sizes in both stages are different than planned. In chapter 4, we review a concrete example from a Likelihood-based clinical trial, and in chapter 5, results of a numerical and theoretical study comparing the Frequentist properties of approaches in the setting where both stages differ \textbf{wording} in different settings are presented.    
%%%%%%%%%%%%%%%%%%%%%%%%%%%%%%%%%%%%%%%%%%%%%%%%%%%%%%%%%%%%%%
%-------------Background-------------------------------------%
%%%%%%%%%%%%%%%%%%%%%%%%%%%%%%%%%%%%%%%%%%%%%%%%%%%%%%%%%%%%%%
\chapter{Background}
\vspace{-7mm}

%------------------------------------------------------------%
%----------- Background on Simon's design -------------------%
%------------------------------------------------------------%
Simon's design will go here. This section will also talk about extending/shortening a trial (unplanned sample sizes) and how recalculating as if it were the planned design will introduce bias and inflate type I error. Talk about prespecifying (maybe here?)

We will only talk about extensions to Simon's design, hypothesis testing, and only stopping for futility in this paper.  

%%%%%%%%%%%%%%%%%%%%%%%%%%%%%%%%%%%%%%%%%%%%%%%%%%%%%%%%%%%%%%
%------------------Literature Review-------------------------%
%%%%%%%%%%%%%%%%%%%%%%%%%%%%%%%%%%%%%%%%%%%%%%%%%%%%%%%%%%%%%%
\chapter{Unplanned Sample Sizes}
This chapter will talk about unplanned sample sizes when only the second stage is different and when both stages can be different. The former will talk about methods such as Koyama and Chen, UMVUE, MLE, etc. The latter will talk about the likelihood design, Chang, adaptation of Chang, and possibly Wu. 

\section{Unplanned Sample Sizes in Second Stage}


\section{Unplanned Sample Sizes in Both Stages}
Decide later if subsections are needed.  \\
\newline

\subsection{\textit{Chang et al.} Alternative Designs and Adaptation}
%%%%%%%%%%%%%%%%%%%%%%%%%%%%%%%%%%%%%%%%%%%%%%%%%%%%%%%%%%%%%%
%--------------------Chang et al method----------------------%
%%%%%%%%%%%%%%%%%%%%%%%%%%%%%%%%%%%%%%%%%%%%%%%%%%%%%%%%%%%%%%

Because accruement of patients can often be unexpected in both stages, it's imperative that methods are available to handle situations with attained sample sizes that differ from the planned sample size. Chang \textit{et al} \cite{Chang} proposed an alternative design that is an extension of Simon’s two stage design to handle unplanned sample sizes in both the first and second stages. Chang \textit{et al} proposes that type II error probability spent in stage I, based on planned and attained sample size, is ... \
\newline
Wu \textit{et al} \cite{Wu} also proposed an adjustment to Simon’s design based on attained sample sizes in both the first and second stages. Because Wu's methods don't work very well \textbf{wording}, we won't consider their method for the remainder of this paper. 



%------------------------------------------------------------%
%----------- Background on likelihood design ----------------%
%------------------------------------------------------------%
\subsection{Likelihood Design}
%\vspace{-5mm}

\noindent Likelihood characteristics:

\vspace{3mm}
%\begin{flushleft}
%\begin{equation*}
$\begin{aligned}
  &LR_n = \frac{L_n(\theta_1)}{L_n(\theta_0)} \in \left\{[0, 1/k], [1/k,k], [k, \infty)\right\} \\
  &\text{Probability of Weak Evidence} \\
  &\indent \gamma_p = P(k_a \leq LR_n \leq K_b | H_p), k_a \leq k_b \\
  &\text{Probability of Strong Evidence} \\
  &\indent \eta_1 = P(LR_n > k_b | H_1) \\
  &\indent \eta_1 = P(LR_n < k_a | H_0) \\
  &\text{Probability of Observing Misleading Evidence}\\
  &\indent \tau_0 = P(LR_n > k_b | H_0), \tau_0 \leq 1/k_b \\
  &\indent \tau_1 = P(LR_n < k_a | H_1), \tau_1 \leq k_a \\
  &\indent \tau_i = O_p(n^{-1/2}) \text{instead of remaining fixed like type I error}
\end{aligned}$

\newpage 

\noindent \textbf{Translating likelihood properties into Simon-like design:} \\
Interim: Translating to successes. This is the region in which we move to stage 2\\

$\begin{aligned}
  & UB_{interim} = \frac{log(k_{bi}) - n_1 log(\frac{1-p_1}{1-p_0})}{log(\frac{p_1(1-p_0)}{p_0(1-p_1)})} \\
  & LB_{interim} = \frac{log(k_{ai}) - n_1 log(\frac{1-p_1}{1-p_0})}{log(\frac{p_1(1-p_0)}{p_0(1-p_1)})} \\
  &\text{(LB, UB) is the interval for weak evidence. If this was Simon's design, } LB_{interim} = r_1 \\
  &\text{Probability of strong, misleading, and weak evidence under the null} \\
  &\indent P(\mbox{Strong}_{0i}) = B(\floor*{LB_{interim}}, n_1, p_0) \\
  &\indent P(Misleading_{0i}) = 1-B(\floor*{UB_{interim}}, n_1, p_0) \\
  &\indent P(Weak_{0i}) = B(\floor*{UB_{interim}}, n_1, p_0) - B(\floor*{LB_{interim}}, n_1, p_0)\\
  &\text{Probability of strong, misleading, and weak evidence under the alternative} \\
  &\indent P(Strong_{0i}) = 1-B(\floor*{UB_{interim}}, n_1, p_1) \\
  &\indent P(Misleading_{0i}) = B(\floor*{LB_{interim}}, n_1, p_1) \\
  &\indent P(Weak_{0i}) = B(\floor*{UB_{interim}}, n_1, p_1) - B(\floor*{LB_{interim}}, n_1, p_1) \\
  &\text{note: under Simon's, PET = 1-P(Weak)}
\end{aligned}$

\newpage 
\vspace{5mm}
\noindent \textbf{Translating likelihood properties into Simon-like design:} \\
Final Stage: Translating to successes.\\
$\begin{aligned}
  & \text{The amount of successes that allow for continuation to the second stage are:  }  \\                   
  & \qquad  (\floor*{LB_{interim}+1}, \floor*{min(n_1, UB_{interim})}) \\
  &\text{Probability of strong, misleading, and weak evidence under } H_p \\
  & P(Weak_p) = \sum_{x=\floor*{LB_{interim}+1}}^{\floor*{min(n_1, UB_{interim})}} \Big(b(x, n_1, p_p) \times B(UB_{interim} - x, n - n_1, p_p)\Big) - B(LB_{interim} - x, n - n_1, p_p)\\
  & P(Strong_p) = P(Strong_{0i}) + \sum_{x=\floor*{LB_{interim}+1}}^{\floor*{min(n_1, UB_{interim})}} \Big( b(x,n_1,p_0) \times B(LB_{interim} - x, n-n_1, p_0) \Big) \\
  & P(Misleading_p) = P(Misleading_{0i}) + \sum_{x=\floor*{LB_{interim}+1}}^{\floor*{min(n_1, UB_{interim})}} \Big(b(x, n_1, p_p) \times (1-B(UB_{interim} - x, n-n_1, p_p) \Big)
\end{aligned}$

\vspace{10mm}
\noindent If we want to translate likelihood design into a Simon's design, we overwrite the LR limits above as:

$\begin{aligned}
 &k_{ai} = OR^{r_1} \frac{1-p_1}{1-p_0}^{n_1} = \frac{1-p_0}{1-p_1}^{r_1-n_1}\frac{p_1}{p_0}^{r_1} \\
 &k_a  = OR^{r} \frac{1-p_1}{1-p_0}^{n} = \frac{1-p_0}{1-p_1}^{r-n}\frac{p_1}{p_0}^{r}\\
 &k_{bi} = k_b = \infty\\
\end{aligned}$

%Note: Better notation, especially for $h_i$ and interim
%\end{equation*}
%\end{flushleft}

\chapter{Example}
Here put the results of comparing the Chang et al paper and adaptation to the protocol of the study. 
We used Monte Carlo simulation to examine the performance of the study design of Chang et al...

\chapter{Results}
The findings will be put here - primarily tables covering different combinations of Simon's designs and unplanned sample sizes. 
\newline
Should maybe talk about how I couldn't replicate some results in the paper. 



%%%%%%%%%%%%%%%%%%%%%%%%%%%%%%%%%%%%%%%%%%%%%%%%%%%%%%%%%%%%%%
%------------------------------------------------%
%%%%%%%%%%%%%%%%%%%%%%%%%%%%%%%%%%%%%%%%%%%%%%%%%%%%%%%%%%%%%%




%%%%%%%%%%%%%%%%%%%%%%%%%%%%%%%%%%%%%%%%%%%%%%%%%%%%%%%%%%%%%%
%------------------------------------------------%
%%%%%%%%%%%%%%%%%%%%%%%%%%%%%%%%%%%%%%%%%%%%%%%%%%%%%%%%%%%%%%





%%%%%%%%%%%%%%%%%%%%%%%%%%%%%%%%%%%%%%%%%%%%%%%%%%%%%%%%%%%%%%
%------------------------------------------------%
%%%%%%%%%%%%%%%%%%%%%%%%%%%%%%%%%%%%%%%%%%%%%%%%%%%%%%%%%%%%%%





%%%%%%%%%%%%%%%%%%%%%%%%%%%%%%%%%%%%%%%%%%%%%%%%%%%%%%%%%%%%%%
%------------------------------------------------%
%%%%%%%%%%%%%%%%%%%%%%%%%%%%%%%%%%%%%%%%%%%%%%%%%%%%%%%%%%%%%%

%%%%%%%%%%%%%%%%%%%%%%%%%%%%%%%%%%%%%%%%%%%%%%%%%%%%%%%%%%%%%%
%-------------- BIBLIOGRAPHY--------------------------------------%
%%%%%%%%%%%%%%%%%%%%%%%%%%%%%%%%%%%%%%%%%%%%%%%%%%%%%%%%%%%%%%

\cite{Koyama}
%\citep{Jung}
%\nocite{Porcher}
%\bibliographystyle{apacite}
\bibliographystyle{ieeetr}
\bibliography{/Users/mollyolson/Documents/Vanderbilt/Masters_Thesis/ThesisWork/thesisBib}		
%\printbibliography

\end{document} 


